\documentclass{beamer}

\usepackage[utf8]{inputenc}
\usepackage[spanish]{babel}
\usepackage{multicol}
\usepackage{graphicx}
\usepackage{hyperref}

\setbeamertemplate{navigation symbols}{}
\usetheme{Montpellier}
\usecolortheme{crane}
\setbeamertemplate{itemize subitem}[circle]
\beamersetuncovermixins{\opaqueness<1>{25}}{\opaqueness<2->{15}}

%Información para la portada
\title{Sensibilización SwL}
\author{Manuel Vidal García\\ Pablo \\ Miguel Rodríguez-Segade}
\date{\today}

\begin{document}

\begin{frame}
    \titlepage
\end{frame}

\begin{frame} \frametitle{Table of contents}
    \tableofcontents
\end{frame}


\section{Introducción}

\begin{frame}\frametitle{SwL}
    ¿Qué es eso del Software Libre?
\end{frame}


\begin{frame}\frametitle{SwL}

    Los usuarios tienen la libertad de ejecutar, copiar, distribuir, estudiar,
    modificar y mejorar el software.


\end{frame}

\begin{frame}\frametitle{Libertades del software libre}

    \begin{itemize}
        \item Ejecutar el programa con cualquier propósito
        \item Estudiar como funciona y cambiarlo para que haga lo que quieras
        \item Redistribuir copias para ayudar al prójimo
        \item Redistribuir copias de sus versiones modificadas
    \end{itemize}

\end{frame}

\begin{frame}\frametitle{Libertades del software libre}

    \begin{block}{¿De que me valen las libertades 1 y 3 si no se programar?}
        \begin{itemize}
            \item Puedes contratar a alguién para que lo cambie
            \item Te garantiza que el programa no hace ninguna perrería
            \item Control colectivo del software
        \end{itemize}
    \end{block}
\end{frame}

\begin{frame}\frametitle{Software Propietario}

    \begin{itemize}
        \item El usuario no sabe lo que esta haciendo el programa
        \item El usuario no es el dueño del programa
        \item El dueño del programa puede ejercer control sobre el usuario
        \item No se puede compartir libremente
        \item Entorpece el avance técnico
        \item Abusa de la ignorancia de los usuarios
    \end{itemize}

    También conocido como ''software sometedor del usuario'' o ''software no
    libre''
    \\
    O el usuario controla el software o el software controla el usuario

\end{frame}


\begin{frame}{Aclaraciones}
    \begin{center}
        Gratis no significa libre     \\~\\

        El software libre no suele ser gratis desarrollarlo\\~\\

        Se puede obtener beneficio económico desarrollando software libre
        (no tanto como con software propietario)
        \\~\\
        Open source no es lo mismo que free software
    \end{center}
\end{frame}

\begin{frame}\frametitle{SwL}
    \begin{block}{¿Por qué es esto tan importante?}
        \begin{itemize}
                \item Sociedad cada vez más dominada por la tecnología
                \item Promueve el desarrollo y avance tecnológico
                \item Defiende la libertad
        \end{itemize}
    \end{block}
\end{frame}

\section{Privacidad}

\begin{frame}
	\center{¿Es importante tu privacidad?}
\end{frame}

\begin{frame}\frametitle{¿Peligra nuestra privacidad?}
	\begin{itemize}
        \item Aumento del volumen de información personal en formato digital
		\item Filtraciones de Snowden - PRISM, XKeyscore
		\item Términos y condiciones de uso
	\end{itemize}

	        \url{https://www.google.com/intl/es/policies/privacy/}
\end{frame}

\begin{frame}\frametitle{Herramientas}

    \center{La privacidad es imposible sin software libre}
    \\~\\
    \url{https://www.eff.org/deeplinks/2014/10/7-privacy-tools-essential-making-citizenfour}
\end{frame}


\section{Licencias}

\begin{frame}\frametitle{Tipos de licencias}
    \begin{columns}[c]
    \column{0.33\textwidth}
    \textbf{Copyleft} \\
    GPL \\ LGPL \\ AGPL
    \column{0.33\textwidth}
    \textbf{Sin copyleft} \\
    BSD \\ Apache \\ MIT \\ MPL
    \column{0.33\textwidth}
    \textbf{Propietarias}
    EULA \\ Freeware \\ Crippelware \\ Proprietary commercial
    \end{columns}
    \begin{center}
        Creative commons licencse
    \end{center}
\end{frame}

\section{Búsqueda de SwL}

\begin{frame}\frametitle{¿Cómo saber si un programa es libre?}
    \begin{itemize}
        \item Buscar licencia página web
        \item Licencia en la Wikipedia
        \item Free software directory
        \item Aceptar condiciones al descargar $\Rightarrow$ mala señal
        \item Página .org $\Rightarrow$ buena señal
    \end{itemize}
\end{frame}

\begin{frame}\frametitle{¿Dónde buscar SwL?}
    \begin{itemize}
        \item alternatives.to
        \item Free software directory
        \item github
        \item sourceforge
        \item Osalt
        \item Repositorios de las distribuciones
        \item Buscador internet
    \end{itemize}
\end{frame}

\begin{frame}\frametitle{Proyectos libres más conocidos}
    \begin{columns}[c]
    \column{0.33\textwidth}
    \textbf{Firefox} \\~\\
    \textbf{VLC} 
    \column{0.33\textwidth}
    \textbf{Linux Kernel} \\~\\
    \textbf{LibreOffice} 
    \column{0.33\textwidth}
    \textbf{Android} \\~\\
    \textbf{Wordpress} 
    \end{columns}
\end{frame}

\section{Otros beneficios SwL}

\begin{frame}\frametitle{Beneficios Personales}
    \begin{itemize}
        \item Ausencia casi total de vulnerabilidades y virus
        \item Más estable
        \item Gratis
        \item No hay crapware
        \item Olvídate de los drivers
        \item Revive ordenadores antiguos
        \item Customización
        \item Estilo Unix
        \item Mejor técnicamente \\
            Mejor rendimiento \\ 
            Mejor sistema de archivos \\
            \ldots 
    \end{itemize}
\end{frame}

\begin{frame}\frametitle{Beneficios Personales}
    \begin{itemize}
        \item Única manera de aprender informática 
        \item Más poder sobre tu sistema
        \item Correción de bugs más rápida y eficaz
        \item Actualización de software centralizada
        \item Instalación de software centralizada
        \item Gran soporte de la comunidad
    \end{itemize}
\end{frame}

\begin{frame}\frametitle{Otros Beneficios}
    \begin{block}{Beneficios en la enseñanza}
        \begin{itemize}
            \item Más educativo
            \item No oculta nada a los alumnos
            \item No genera dependencia en software propietario
        \end{itemize}
    \end{block}

    \begin{block}{Beneficios en la administración}
        \begin{itemize}
            \item Reducción de costes
            \item Independencia de empresas privadas
        \end{itemize}
    \end{block}

    \begin{block}{Beneficios en cooperación}
        \begin{itemize}
            \item Reduce la desigualdad y brecha digital
        \end{itemize}
    \end{block}
\end{frame}

\section{Mitos}

\begin{frame}
    \begin{center}
        \textbf{Es más dificil de usar}
    \end{center}
\end{frame}

\begin{frame}
    \begin{center}
        \textbf{Es de peor calidad}
    \end{center}
\end{frame}

\begin{frame}
    \begin{center}
        \textbf{Descargando windows sin licencia no contribuyo al software propietario}
    \end{center}
\end{frame}

\begin{frame}
    \begin{center}
        \textbf{Para hacer trabajo serio no se usa software libre}
    \end{center}
\end{frame}

\section{Otros temas}

\begin{frame}\frametitle{Purismo}
    \begin{itemize}
        \item firmware/BIOS/drivers
        \item javascript no-libre
        \item formatos no libres
        \item SaaS
        \item Distribuciones 100\% libres
    \end{itemize}
\end{frame}

\begin{frame}\frametitle{Otras iniciativas similares}
    \begin{itemize}
        \item Open access
        \item Open hardware
        \item Wikimedia
    \end{itemize}
\end{frame}

\begin{frame}\frametitle{Organizaciones}
    \begin{itemize}
        \item eff
        \item fsf
    \end{itemize}
\end{frame}

\begin{frame}\frametitle{Historia}
    \begin{itemize}
        \item 50-70s Todo se desarrollaba de manera colaborativa \\
              Se solía dar el código junto con el software
        \item 1976 Bill Gates escribe ''Open letter to hobbyists''
        \item 1983 Richard Stallman empieza el Proyecto GNU
        \item 1992 Linus Torvalds publica el kernel linux bajó licencia GPL
        \item 1997 Movimiento ''Open Source''
    \end{itemize}
\end{frame}

\end{document}

\documentclass{beamer}

\usepackage[utf8]{inputenc}
\usepackage[spanish]{babel}
\usepackage{multicol}
\usepackage{graphicx}
\usepackage{hyperref}

\setbeamertemplate{navigation symbols}{}
\usetheme{Montpellier}
\usecolortheme{crane}
\setbeamertemplate{itemize subitem}[circle]
\beamersetuncovermixins{\opaqueness<1>{25}}{\opaqueness<2->{15}}

%Información para la portada
\title{Sensibilización SwL}
\author{Manolo el del bombo\\Miguel el tocabolas}
\date{\today}

\begin{document}

\begin{frame}
    \titlepage
\end{frame}

\begin{frame} \frametitle{Table of contents}
    \tableofcontents
\end{frame}


\section{¿Por qué?}

\begin{frame}\frametitle{SwL}
    ¿Qué es eso del Software Libre?
\end{frame}


\begin{frame}\frametitle{SwL}

    Software: \pause Conjunto de programas,
    instrucciones y reglas informáticas para
    ejecutar ciertas tareas en una computadora.

\end{frame}

\begin{frame}\frametitle{SwL}

    \includegraphics[scale=0.1]{figures/The_GNU_logo.png}
    
\end{frame}

\begin{frame}\frametitle{Privacidad}

	\center{¿Es importante tu privacidad?}

\end{frame}

\begin{frame}\frametitle{¿Peligra nuestra privacidad?}

	\begin{itemize}
		\item Filtraciones de Snowden
		\item Software propietario
		\item Términos y condiciones de uso
	\end{itemize}
\end{frame}

\begin{frame}
	\url{https://www.google.com/intl/es/policies/privacy/}
\end{frame}


\end{document}
